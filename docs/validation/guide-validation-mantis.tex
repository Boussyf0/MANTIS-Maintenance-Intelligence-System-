\documentclass[12pt,a4paper]{article}

% Packages
\usepackage[utf8]{inputenc}
\usepackage[french]{babel}
\usepackage[T1]{fontenc}
\usepackage{geometry}
\usepackage{graphicx}
\usepackage{xcolor}
\usepackage{listings}
\usepackage{tcolorbox}
\usepackage{hyperref}
\usepackage{fancyhdr}
\usepackage{tikz}
\usepackage{enumitem}
\usepackage{tabularx}
\usepackage{booktabs}
\usepackage{fontawesome5}
\usepackage{tocloft}

% Configuration de la géométrie
\geometry{
    left=2.5cm,
    right=2.5cm,
    top=3cm,
    bottom=3cm
}

% Configuration des liens
\hypersetup{
    colorlinks=true,
    linkcolor=blue,
    filecolor=magenta,
    urlcolor=cyan,
    pdftitle={Guide de Validation MANTIS},
    pdfauthor={MANTIS Team - EMSI},
}

% Configuration des listings de code
\lstdefinestyle{bash}{
    language=bash,
    basicstyle=\ttfamily\small,
    backgroundcolor=\color{gray!10},
    keywordstyle=\color{blue}\bfseries,
    stringstyle=\color{red},
    commentstyle=\color{green!60!black},
    showstringspaces=false,
    breaklines=true,
    frame=single,
    rulecolor=\color{gray!30},
    numbers=left,
    numberstyle=\tiny\color{gray},
}

\lstdefinestyle{git}{
    basicstyle=\ttfamily\small,
    backgroundcolor=\color{blue!5},
    frame=single,
    rulecolor=\color{blue!30},
    breaklines=true,
}

% Configuration des boîtes
\tcbuselibrary{skins,breakable}

\newtcolorbox{infobox}[1][]{
    colback=blue!5!white,
    colframe=blue!75!black,
    fonttitle=\bfseries,
    title=#1,
    breakable
}

\newtcolorbox{warningbox}[1][]{
    colback=orange!5!white,
    colframe=orange!75!black,
    fonttitle=\bfseries,
    title=#1,
    breakable
}

\newtcolorbox{successbox}[1][]{
    colback=green!5!white,
    colframe=green!75!black,
    fonttitle=\bfseries,
    title=#1,
    breakable
}

\newtcolorbox{errorbox}[1][]{
    colback=red!5!white,
    colframe=red!75!black,
    fonttitle=\bfseries,
    title=#1,
    breakable
}

% En-têtes et pieds de page
\pagestyle{fancy}
\fancyhf{}
\fancyhead[L]{\leftmark}
\fancyhead[R]{\includegraphics[height=1cm]{logo.png}}
\fancyfoot[C]{\thepage}
\renewcommand{\headrulewidth}{0.4pt}
\renewcommand{\footrulewidth}{0.4pt}

% Commandes personnalisées
\newcommand{\mantis}{\textbf{MANTIS}}
\newcommand{\code}[1]{\texttt{#1}}
\newcommand{\important}[1]{\textcolor{red}{\textbf{#1}}}
\newcommand{\success}[1]{\textcolor{green!60!black}{\textbf{#1}}}
\newcommand{\warning}[1]{\textcolor{orange}{\textbf{#1}}}

\begin{document}

% Page de titre
\begin{titlepage}
    \centering
    \vspace*{2cm}

    {\Huge\bfseries Guide du Système de Validation\par}
    \vspace{0.5cm}
    {\Large\bfseries MANTIS\par}
    \vspace{0.3cm}
    {\large MAiNtenance prédictive Temps-réel pour usines Intelligentes\par}

    \vspace{2cm}

    \begin{tikzpicture}
        \draw[blue!50, line width=3pt] (0,0) circle (3cm);
        \node[align=center] at (0,0) {\Huge\faCheckCircle[regular]};
        \node[align=center] at (0,-1.5) {\Large Qualité};
        \node[align=center] at (0,-2) {\Large Garantie};
    \end{tikzpicture}

    \vspace{2cm}

    {\Large Guide Complet par Profil\par}

    \vspace{1cm}

    \begin{tabular}{rl}
        \faUsers & Développeurs Backend/Frontend \\
        \faCog & DevOps \& Administrateurs \\
        \faChartLine & Chefs de Projet \\
        \faGraduationCap & Data Scientists \\
    \end{tabular}

    \vfill

    {\large EMSI Engineering School\par}
    {\large Version 1.0 -- \today\par}
\end{titlepage}

% Table des matières
\tableofcontents
\newpage

% Introduction
\section{Introduction}

\subsection{Vue d'ensemble}

Le système de validation \mantis{} garantit la qualité du code à travers trois niveaux de protection :

\begin{enumerate}[label=\textbf{\arabic*.}]
    \item \textbf{Hooks Git locaux} -- Validation immédiate avant commit/push
    \item \textbf{GitHub Actions CI/CD} -- Validation automatique sur le serveur
    \item \textbf{Scripts de validation} -- Validation manuelle complète
\end{enumerate}

\begin{successbox}[Objectifs du système]
\begin{itemize}
    \item \success{✓} Garantir 100\% des commits avec format standardisé
    \item \success{✓} Bloquer les fichiers sensibles (credentials, secrets)
    \item \success{✓} Assurer une couverture de tests ≥ 80\%
    \item \success{✓} Maintenir un code propre et maintenable
    \item \success{✓} Traçabilité complète des changements
\end{itemize}
\end{successbox}

\subsection{Architecture du système}

\begin{figure}[h]
\centering
\begin{tikzpicture}[node distance=2cm, auto]
    % Nœuds
    \node[draw, rectangle, fill=blue!20, minimum width=3cm, minimum height=1cm] (dev) {Développeur};
    \node[draw, rectangle, fill=green!20, minimum width=3cm, minimum height=1cm, below of=dev] (hooks) {Git Hooks};
    \node[draw, rectangle, fill=orange!20, minimum width=3cm, minimum height=1cm, below of=hooks] (local) {Tests Locaux};
    \node[draw, rectangle, fill=purple!20, minimum width=3cm, minimum height=1cm, below of=local] (ci) {GitHub Actions};
    \node[draw, rectangle, fill=red!20, minimum width=3cm, minimum height=1cm, below of=ci] (deploy) {Déploiement};

    % Flèches
    \draw[->] (dev) -- node[right] {git commit} (hooks);
    \draw[->] (hooks) -- node[right] {Validation} (local);
    \draw[->] (local) -- node[right] {git push} (ci);
    \draw[->] (ci) -- node[right] {Si OK} (deploy);

    % Annotations
    \node[right of=hooks, node distance=5cm, align=left] {pre-commit\\commit-msg};
    \node[right of=local, node distance=5cm, align=left] {Maven\\pytest};
    \node[right of=ci, node distance=5cm, align=left] {Tests\\Security\\Build};
\end{tikzpicture}
\caption{Architecture du système de validation}
\end{figure}

\newpage

\section{Format de Commit Standardisé}

\subsection{Format Conventional Commits}

Tous les commits doivent suivre le format \textbf{Conventional Commits} :

\begin{lstlisting}[style=git]
<type>(<scope>): <description>

<body optionnel>

<footer optionnel>
\end{lstlisting}

\subsection{Types valides}

\begin{table}[h]
\centering
\begin{tabularx}{\textwidth}{|l|X|X|}
\hline
\textbf{Type} & \textbf{Description} & \textbf{Exemple} \\
\hline
\code{feat} & Nouvelle fonctionnalité & \code{feat(ingestion): ajouter support Modbus TCP} \\
\hline
\code{fix} & Correction de bug & \code{fix(rul): corriger prédiction RUL < 24h} \\
\hline
\code{docs} & Documentation uniquement & \code{docs(readme): mettre à jour installation} \\
\hline
\code{style} & Formatage, pas de code & \code{style(preprocessing): formater selon PEP8} \\
\hline
\code{refactor} & Refactoring & \code{refactor(features): optimiser calcul FFT} \\
\hline
\code{perf} & Amélioration performance & \code{perf(anomaly): réduire latence de 30\%} \\
\hline
\code{test} & Ajout/modification tests & \code{test(rul): ajouter tests LSTM} \\
\hline
\code{build} & Système de build & \code{build(docker): optimiser image} \\
\hline
\code{ci} & Configuration CI/CD & \code{ci(actions): ajouter cache Maven} \\
\hline
\code{chore} & Tâches de maintenance & \code{chore(deps): mettre à jour PyTorch} \\
\hline
\end{tabularx}
\caption{Types de commits valides}
\end{table}

\subsection{Scopes suggérés}

\begin{itemize}[label=\faCircle]
    \item \code{ingestion} -- Service d'ingestion IIoT
    \item \code{preprocessing} -- Service de prétraitement
    \item \code{features} -- Service d'extraction de features
    \item \code{anomaly} -- Service de détection d'anomalies
    \item \code{rul} -- Service de prédiction RUL
    \item \code{orchestrator} -- Service orchestrateur
    \item \code{dashboard} -- Dashboard React
    \item \code{infrastructure} -- Docker, Kubernetes
    \item \code{database} -- Schémas, migrations
    \item \code{docs} -- Documentation
    \item \code{tests} -- Tests
\end{itemize}

\subsection{Exemples}

\begin{successbox}[Messages VALIDES ✓]
\begin{lstlisting}[style=git]
feat(ingestion): ajouter support pour Modbus TCP
fix(rul): corriger prediction pour RUL < 24h
docs(readme): mettre a jour les instructions d'installation
refactor(preprocessing): optimiser le pipeline de nettoyage
perf(features): reduire temps de calcul FFT de 30%
test(anomaly): ajouter tests unitaires pour Isolation Forest
\end{lstlisting}
\end{successbox}

\begin{errorbox}[Messages INVALIDES ✗]
\begin{lstlisting}[style=git]
Added new feature              # Pas de type
feat: Added feature.           # Point final interdit
FEAT(ingestion): add support   # Type en majuscule
feat add support               # Manque les deux-points
fix                            # Description trop courte
\end{lstlisting}
\end{errorbox}

\newpage

\section{Guide par Profil}

\subsection{\faCode{} Développeur Backend (Java/Spring Boot)}

\subsubsection{Installation initiale}

\begin{lstlisting}[style=bash]
# 1. Cloner le projet
git clone <repo-url>
cd MANTIS

# 2. Installer les hooks Git
./scripts/install-hooks.sh

# 3. Configurer le template de commit
git config commit.template .gitmessage

# 4. Verifier l'installation
git config core.hooksPath  # Devrait afficher: .githooks
\end{lstlisting}

\subsubsection{Workflow quotidien}

\begin{enumerate}
    \item \textbf{Créer une branche}
    \begin{lstlisting}[style=bash]
git checkout -b feature/ma-fonctionnalite
    \end{lstlisting}

    \item \textbf{Développer} (exemple : service Java)
    \begin{lstlisting}[style=bash]
cd services/ingestion-iiot
# ... modifications du code ...
    \end{lstlisting}

    \item \textbf{Tester localement}
    \begin{lstlisting}[style=bash]
mvn clean test
mvn jacoco:report  # Verifier la couverture
    \end{lstlisting}

    \item \textbf{Commit} (le hook pre-commit validera automatiquement)
    \begin{lstlisting}[style=bash]
git add .
git commit
# Ou directement avec message
git commit -m "feat(ingestion): ajouter support Modbus TCP"
    \end{lstlisting}

    \item \textbf{Push} (le hook pre-push executera les tests)
    \begin{lstlisting}[style=bash]
git push origin feature/ma-fonctionnalite
    \end{lstlisting}
\end{enumerate}

\subsubsection{Ce qui est validé automatiquement}

\begin{infobox}[Validations pour code Java]
\begin{itemize}
    \item \faCheckCircle{} Compilation Maven réussie
    \item \faCheckCircle{} Tests unitaires passés (JUnit 5)
    \item \faCheckCircle{} Couverture de code ≥ 80\% (JaCoCo)
    \item \faCheckCircle{} Pas de fichiers sensibles (\code{.env}, credentials)
    \item \faCheckCircle{} Format de commit valide
\end{itemize}
\end{infobox}

\subsubsection{Commandes utiles}

\begin{lstlisting}[style=bash]
# Compiler uniquement
mvn clean compile

# Tester un seul test
mvn test -Dtest=ClassName#methodName

# Rapport de couverture
mvn jacoco:report
# Voir: target/site/jacoco/index.html

# Build complet
mvn clean install
\end{lstlisting}

\subsubsection{Dépannage}

\begin{warningbox}[Tests Java échouent]
\begin{lstlisting}[style=bash]
# Mode debug
mvn clean test -X

# Skip tests temporairement (deconseille)
mvn clean install -DskipTests

# Verifier les dependances
mvn dependency:tree
\end{lstlisting}
\end{warningbox}

\newpage

\subsection{\faPython{} Data Scientist / Développeur ML (Python)}

\subsubsection{Installation initiale}

\begin{lstlisting}[style=bash]
# 1. Cloner et configurer
git clone <repo-url>
cd MANTIS

# 2. Installer les hooks
./scripts/install-hooks.sh

# 3. Creer environnement virtuel
python -m venv venv
source venv/bin/activate  # Linux/Mac
# venv\Scripts\activate   # Windows

# 4. Installer dependances
pip install -r requirements-dev.txt

# 5. Installer les outils de validation
pip install flake8 black pytest pytest-cov
\end{lstlisting}

\subsubsection{Workflow quotidien}

\begin{enumerate}
    \item \textbf{Créer une branche}
    \begin{lstlisting}[style=bash]
git checkout -b feature/nouveau-modele-rul
    \end{lstlisting}

    \item \textbf{Développer} (exemple : modèle RUL)
    \begin{lstlisting}[style=bash]
cd services/rul-prediction
# ... developper le modele ...
    \end{lstlisting}

    \item \textbf{Formater le code}
    \begin{lstlisting}[style=bash]
black . --line-length=120
isort . --profile black
    \end{lstlisting}

    \item \textbf{Vérifier le linting}
    \begin{lstlisting}[style=bash]
flake8 . --max-line-length=120
    \end{lstlisting}

    \item \textbf{Tester}
    \begin{lstlisting}[style=bash]
pytest tests/ -v
pytest tests/ --cov=src --cov-report=html
# Voir: htmlcov/index.html
    \end{lstlisting}

    \item \textbf{Commit et push}
    \begin{lstlisting}[style=bash]
git add .
git commit -m "feat(rul): ajouter modele TCN pour prediction"
git push origin feature/nouveau-modele-rul
    \end{lstlisting}
\end{enumerate}

\subsubsection{Ce qui est validé automatiquement}

\begin{infobox}[Validations pour code Python]
\begin{itemize}
    \item \faCheckCircle{} Formatage Black (line-length 120)
    \item \faCheckCircle{} Linting flake8
    \item \faCheckCircle{} Tests pytest passés
    \item \faCheckCircle{} Couverture ≥ 80\%
    \item \faCheckCircle{} Import ordering (isort)
    \item \faCheckCircle{} Pas de fichiers sensibles
\end{itemize}
\end{infobox}

\subsubsection{Bonnes pratiques ML}

\begin{successbox}[Enregistrement des modèles avec MLflow]
\begin{lstlisting}[language=Python, basicstyle=\ttfamily\small]
import mlflow
import mlflow.pytorch

with mlflow.start_run():
    # Enregistrer les parametres
    mlflow.log_params({
        "model": "lstm",
        "hidden_size": 50,
        "num_layers": 2
    })

    # Entrainer le modele
    model.fit(X_train, y_train)

    # Enregistrer les metriques
    mlflow.log_metrics({
        "rmse": 12.5,
        "mae": 8.3,
        "r2": 0.92
    })

    # Enregistrer le modele
    mlflow.pytorch.log_model(model, "rul_model")
\end{lstlisting}
\end{successbox}

\newpage

\subsection{\faReact{} Développeur Frontend (React/Next.js)}

\subsubsection{Installation initiale}

\begin{lstlisting}[style=bash]
# 1. Cloner et configurer
git clone <repo-url>
cd MANTIS

# 2. Installer les hooks
./scripts/install-hooks.sh

# 3. Installer dependances frontend
cd services/dashboard
npm install
\end{lstlisting}

\subsubsection{Workflow quotidien}

\begin{enumerate}
    \item \textbf{Développer le dashboard}
    \begin{lstlisting}[style=bash]
cd services/dashboard
npm run dev  # Demarrer en mode dev
    \end{lstlisting}

    \item \textbf{Formater le code}
    \begin{lstlisting}[style=bash]
npm run format  # Prettier
npm run lint    # ESLint
    \end{lstlisting}

    \item \textbf{Tester}
    \begin{lstlisting}[style=bash]
npm run test
npm run test:coverage
    \end{lstlisting}

    \item \textbf{Build}
    \begin{lstlisting}[style=bash]
npm run build
    \end{lstlisting}

    \item \textbf{Commit}
    \begin{lstlisting}[style=bash]
git add .
git commit -m "feat(dashboard): ajouter graphe temps reel RUL"
git push
    \end{lstlisting}
\end{enumerate}

\subsubsection{Structure des composants}

\begin{lstlisting}[language=JavaScript, basicstyle=\ttfamily\small]
// services/dashboard/src/components/RULChart.tsx
import React from 'react';
import { Line } from 'recharts';

export const RULChart: React.FC<RULChartProps> = ({ data }) => {
  return (
    <div className="rul-chart">
      <Line data={data} ... />
    </div>
  );
};
\end{lstlisting}

\newpage

\subsection{\faCog{} DevOps / Administrateur Système}

\subsubsection{Installation de l'infrastructure}

\begin{lstlisting}[style=bash]
# 1. Cloner le projet
git clone <repo-url>
cd MANTIS

# 2. Demarrer l'infrastructure
make docker-up
# ou
cd infrastructure/docker
docker-compose -f docker-compose.infrastructure.yml up -d

# 3. Verifier l'etat
docker-compose ps
\end{lstlisting}

\subsubsection{Configuration CI/CD}

Le workflow GitHub Actions est dans \code{.github/workflows/ci.yml}.

\begin{infobox}[Jobs CI/CD automatiques]
\begin{enumerate}
    \item \textbf{validate-commit-messages} -- Vérifie les messages de commit
    \item \textbf{code-quality} -- Scan de qualité du code
    \item \textbf{test-java-services} -- Tests Java (Maven + JaCoCo)
    \item \textbf{test-python-services} -- Tests Python (pytest)
    \item \textbf{integration-tests} -- Tests d'intégration
    \item \textbf{docker-build} -- Build des images Docker
    \item \textbf{security-scan} -- Scan Trivy pour vulnérabilités
    \item \textbf{deployment-ready} -- Vérification finale
\end{enumerate}
\end{infobox}

\subsubsection{Monitoring}

\begin{lstlisting}[style=bash]
# URLs de monitoring
make monitor

# Sortie:
#   Grafana:    http://localhost:3001
#   Prometheus: http://localhost:9090
#   Jaeger:     http://localhost:16686
#   Kafka UI:   http://localhost:8080
#   MLflow:     http://localhost:5000
\end{lstlisting}

\subsubsection{Gestion des secrets}

\begin{warningbox}[IMPORTANT: Secrets]
\textbf{JAMAIS} commiter de secrets dans le repository !

Les hooks Git détectent automatiquement :
\begin{itemize}
    \item Fichiers \code{.env} (sauf \code{.env.example})
    \item \code{credentials*.json}
    \item \code{*.key}, \code{*.pem}
    \item Patterns : \code{password}, \code{secret}, \code{api\_key}
\end{itemize}

Pour la production, utiliser :
\begin{itemize}
    \item HashiCorp Vault
    \item Kubernetes Secrets
    \item Variables d'environnement
\end{itemize}
\end{warningbox}

\subsubsection{Backup et restauration}

\begin{lstlisting}[style=bash]
# Backup
make backup-db

# Restauration
docker exec -i mantis-postgres psql -U mantis -d mantis \
  < backups/postgres_YYYYMMDD_HHMMSS.sql
\end{lstlisting}

\newpage

\subsection{\faChartLine{} Chef de Projet}

\subsubsection{Vue d'ensemble du système de validation}

Le système garantit la qualité à travers :

\begin{table}[h]
\centering
\begin{tabularx}{\textwidth}{|l|X|X|}
\hline
\textbf{Niveau} & \textbf{Quand} & \textbf{Ce qui est vérifié} \\
\hline
Local (Hooks Git) & Avant chaque commit/push & Format commit, tests unitaires, linting \\
\hline
CI/CD (GitHub Actions) & À chaque push/PR & Tests complets, sécurité, build Docker \\
\hline
Manuel (Scripts) & Sur demande & Structure complète du projet \\
\hline
\end{tabularx}
\caption{Niveaux de validation}
\end{table}

\subsubsection{Métriques de qualité}

\begin{successbox}[Garanties du système]
\begin{itemize}
    \item \success{100\%} des commits suivent le format standardisé
    \item \success{0} fichiers sensibles dans le repository
    \item \success{≥80\%} de couverture de code (Java \& Python)
    \item \success{0} tests échoués avant merge
    \item \success{0} vulnérabilités critiques (scan Trivy)
\end{itemize}
\end{successbox}

\subsubsection{Suivi des Pull Requests}

\begin{enumerate}
    \item \textbf{Création de la PR} -- Développeur pousse sa branche
    \item \textbf{CI/CD automatique} -- GitHub Actions s'exécute
    \item \textbf{Revue de code} -- Au moins 1 approbation requise
    \item \textbf{Merge} -- Possible uniquement si tous les checks sont verts
\end{enumerate}

\begin{figure}[h]
\centering
\begin{tikzpicture}[node distance=1.5cm]
    \node[draw, circle, fill=blue!20] (pr) {PR};
    \node[draw, circle, fill=orange!20, below of=pr] (ci) {CI/CD};
    \node[draw, circle, fill=green!20, below left of=ci] (ok) {✓};
    \node[draw, circle, fill=red!20, below right of=ci] (nok) {✗};
    \node[draw, rectangle, fill=green!20, below of=ok] (merge) {Merge};
    \node[draw, rectangle, fill=red!20, below of=nok] (fix) {Fix};

    \draw[->] (pr) -- (ci);
    \draw[->] (ci) -- node[left] {OK} (ok);
    \draw[->] (ci) -- node[right] {Échec} (nok);
    \draw[->] (ok) -- (merge);
    \draw[->] (nok) -- (fix);
    \draw[->, dashed] (fix) -- ++(3,0) |- (pr);
\end{tikzpicture}
\caption{Workflow de Pull Request}
\end{figure}

\subsubsection{Rapports disponibles}

\begin{itemize}
    \item \textbf{Coverage} -- Couverture de code (JaCoCo pour Java, pytest-cov pour Python)
    \item \textbf{Tests} -- Résultats des tests (Surefire reports)
    \item \textbf{Security} -- Vulnérabilités détectées (Trivy)
    \item \textbf{Commits} -- Historique propre avec format standardisé
\end{itemize}

\newpage

\section{Commandes Essentielles}

\subsection{Makefile - Vue d'ensemble}

\begin{lstlisting}[style=bash]
# Voir toutes les commandes disponibles
make help

# === Developpement ===
make install          # Installer dependances Python
make start            # Demarrer infrastructure
make stop             # Arreter services
make clean            # Nettoyer (cache, containers)
make dataset          # Telecharger dataset C-MAPSS
make notebook         # Lancer Jupyter

# === Qualite du code ===
make test             # Lancer tous les tests
make lint             # Verifier le code
make format           # Formater le code
make validate         # Valider structure projet
make install-hooks    # Installer Git hooks
make test-validation  # Demo du systeme

# === Docker ===
make docker-build     # Construire images
make docker-up        # Demarrer Docker Compose
make docker-down      # Arreter Docker Compose
\end{lstlisting}

\subsection{Scripts de validation}

\begin{lstlisting}[style=bash]
# Installer les hooks Git
./scripts/install-hooks.sh

# Valider la structure complete du projet
./scripts/validate-project.sh

# Demonstration du systeme
./scripts/test-validation.sh

# Demarrer tous les services
./scripts/start-services.sh

# Arreter tous les services
./scripts/stop-services.sh

# Telecharger dataset NASA C-MAPSS
./scripts/download-cmapss.sh
\end{lstlisting}

\newpage

\section{Dépannage}

\subsection{Hooks Git ne s'exécutent pas}

\begin{errorbox}[Problème : Hooks ignorés]
\textbf{Symptôme} : Les commits passent sans validation.

\textbf{Solution} :
\begin{lstlisting}[style=bash]
# Verifier la configuration
git config core.hooksPath

# Si vide ou incorrect, reinstaller
./scripts/install-hooks.sh

# Verifier les permissions
chmod +x .githooks/*
ls -la .githooks/
\end{lstlisting}
\end{errorbox}

\subsection{Message de commit rejeté}

\begin{errorbox}[Problème : Format invalide]
\textbf{Symptôme} : \code{✗ Format invalide}

\textbf{Solution} : Utiliser le template
\begin{lstlisting}[style=bash]
# Configurer le template
git config commit.template .gitmessage

# Commiter sans -m pour ouvrir l'editeur
git commit

# Ou verifier le format :
# <type>(<scope>): <description>
# Exemple valide:
git commit -m "feat(ingestion): ajouter support Modbus TCP"
\end{lstlisting}
\end{errorbox}

\subsection{Tests Java échouent}

\begin{errorbox}[Problème : Tests Maven]
\textbf{Symptôme} : \code{✗ Tests échoués pour services/ingestion-iiot}

\textbf{Solution} :
\begin{lstlisting}[style=bash]
cd services/ingestion-iiot

# Mode debug
mvn clean test -X

# Verifier une classe specifique
mvn test -Dtest=ClassName

# Nettoyer et rebuilder
mvn clean install
\end{lstlisting}
\end{errorbox}

\subsection{Tests Python échouent}

\begin{errorbox}[Problème : Tests pytest]
\textbf{Symptôme} : \code{✗ Tests Python échoués}

\textbf{Solution} :
\begin{lstlisting}[style=bash]
# Mode verbose
pytest -v

# Traceback complet
pytest --tb=long

# Tester un fichier specifique
pytest tests/test_specific.py -v

# Verifier la couverture
pytest --cov=src tests/
\end{lstlisting}
\end{errorbox}

\subsection{Fichier sensible détecté}

\begin{errorbox}[Problème : Secret détecté]
\textbf{Symptôme} : \code{✗ Fichier sensible détecté : .env}

\textbf{Solution} :
\begin{lstlisting}[style=bash]
# Retirer le fichier du staging
git reset HEAD .env

# Ajouter au .gitignore si necessaire
echo ".env" >> .gitignore

# Utiliser .env.example pour les templates
cp .env .env.example
# Puis editer .env.example pour retirer les vraies valeurs
\end{lstlisting}
\end{errorbox}

\newpage

\section{FAQ}

\subsection{Questions générales}

\textbf{Q: Puis-je désactiver temporairement les hooks ?}

\textbf{R:} Oui, mais déconseillé :
\begin{lstlisting}[style=bash]
git commit --no-verify
git push --no-verify
\end{lstlisting}

\vspace{0.5cm}

\textbf{Q: Les hooks ralentissent mes commits, que faire ?}

\textbf{R:} Les hooks exécutent uniquement les tests des services modifiés. Pour un commit WIP rapide sur une branche de dev :
\begin{lstlisting}[style=bash]
git commit --no-verify -m "wip: travail en cours"
# Puis avant le push final, executer les tests
make test
\end{lstlisting}

\vspace{0.5cm}

\textbf{Q: Comment tester un hook avant de commiter ?}

\textbf{R:} Exécuter directement :
\begin{lstlisting}[style=bash]
.githooks/pre-commit
.githooks/commit-msg /tmp/test_msg
\end{lstlisting}

\vspace{0.5cm}

\textbf{Q: Les hooks fonctionnent-ils sur Windows ?}

\textbf{R:} Oui, mais nécessite Git Bash ou WSL2 pour exécuter les scripts bash.

\subsection{Questions techniques}

\textbf{Q: Comment ajouter un nouveau type de commit ?}

\textbf{R:} Modifier le hook \code{.githooks/commit-msg} et mettre à jour la variable \code{VALID\_TYPES}.

\vspace{0.5cm}

\textbf{Q: Puis-je personnaliser les validations pour mon service ?}

\textbf{R:} Oui, modifier \code{.githooks/pre-commit} pour ajouter des validations spécifiques.

\vspace{0.5cm}

\textbf{Q: GitHub Actions échoue mais pas les hooks locaux, pourquoi ?}

\textbf{R:} Vérifier les versions de Java/Python/Maven. GitHub Actions utilise les versions spécifiées dans \code{.github/workflows/ci.yml}.

\newpage

\section{Annexes}

\subsection{Annexe A : Structure des fichiers}

\begin{lstlisting}[basicstyle=\ttfamily\footnotesize]
MANTIS/
├── .githooks/               # Hooks Git personnalises
│   ├── pre-commit          # Validation avant commit
│   ├── commit-msg          # Validation message
│   ├── pre-push            # Tests avant push
│   └── README.md           # Documentation hooks
├── .github/
│   └── workflows/
│       └── ci.yml          # Pipeline CI/CD
├── scripts/
│   ├── install-hooks.sh    # Installation hooks
│   ├── validate-project.sh # Validation projet
│   └── test-validation.sh  # Demo systeme
├── docs/
│   └── validation/
│       └── guide-validation-mantis.tex  # Ce document
├── .gitmessage             # Template commit
├── README_VALIDATION.md    # Guide validation
├── VALIDATION_SETUP_SUMMARY.md
└── CONTRIBUTING.md         # Guide contribution
\end{lstlisting}

\subsection{Annexe B : Références}

\begin{itemize}
    \item \textbf{Conventional Commits} : \url{https://www.conventionalcommits.org/}
    \item \textbf{Git Hooks} : \url{https://git-scm.com/book/en/v2/Customizing-Git-Git-Hooks}
    \item \textbf{GitHub Actions} : \url{https://docs.github.com/en/actions}
    \item \textbf{Maven Surefire} : \url{https://maven.apache.org/surefire/}
    \item \textbf{JaCoCo} : \url{https://www.jacoco.org/}
    \item \textbf{pytest} : \url{https://docs.pytest.org/}
\end{itemize}

\subsection{Annexe C : Contacts}

\begin{table}[h]
\centering
\begin{tabularx}{\textwidth}{|l|X|}
\hline
\textbf{Email} & O.ouedrhiri@emsi.ma, H.Tabbaa@emsi.ma, lachgar.m@gmail.com \\
\hline
\textbf{GitHub} & \url{https://github.com/MANTIS} \\
\hline
\textbf{Documentation} & \code{docs/} \\
\hline
\textbf{Issues} & GitHub Issues \\
\hline
\textbf{Discussions} & GitHub Discussions \\
\hline
\end{tabularx}
\caption{Contacts et ressources}
\end{table}

\vfill

\begin{center}
\begin{tikzpicture}
    \draw[blue!50, line width=2pt] (0,0) rectangle (10,2);
    \node[align=center] at (5,1) {
        \Large\textbf{Système de Validation MANTIS} \\
        \large Qualité Garantie à Chaque Push ✓
    };
\end{tikzpicture}
\end{center}

\end{document}
