\documentclass[a4paper,12pt]{report}
\usepackage[utf8]{inputenc}
\usepackage[T1]{fontenc}
\usepackage[french]{babel}
\usepackage{geometry}
\usepackage{graphicx}
\usepackage{xcolor}
\usepackage{tikz}
\usepackage{float}
\usepackage{hyperref}
\usepackage{listings}
\usepackage{booktabs}
\usepackage{titlesec}
\usepackage{fancyhdr}
\usepackage{enumitem}

% Configuration Geometry
\geometry{hmargin=2.5cm,vmargin=2.5cm}

% Configuration TikZ
\usetikzlibrary{shapes.geometric, arrows.meta, positioning, calc, fit, backgrounds, shadows}

% Couleurs MANTIS
\definecolor{mantisblue}{RGB}{0, 102, 204}
\definecolor{mantisgreen}{RGB}{0, 153, 76}
\definecolor{mantisorange}{RGB}{255, 128, 0}
\definecolor{mantisred}{RGB}{204, 0, 0}
\definecolor{mantisgray}{RGB}{128, 128, 128}

% Configuration Hyperref
\hypersetup{
    colorlinks=true,
    linkcolor=mantisblue,
    filecolor=magenta,      
    urlcolor=cyan,
}

% Header/Footer
\pagestyle{fancy}
\fancyhf{}
\lhead{\textbf{MANTIS}}
\rhead{Rapport de Conception et Architecture}
\cfoot{\thepage}

\title{\textbf{\Huge MANTIS}\\ \Large MAiNtenance prédictive Temps-réel pour usines Intelligentes\\ \vspace{1cm} \textbf{Dossier de Conception et Architecture}}
\author{Équipe MANTIS}
\date{\today}

\begin{document}

\maketitle

\tableofcontents
\newpage

%=============================================================================
% CHAPITRE 1 : PROCESSUS MÉTIERS (BPMN)
%=============================================================================
\chapter{Processus Métiers (BPMN)}

\section{Introduction}
Ce chapitre décrit les processus métiers principaux de la plateforme MANTIS, modélisés à l'aide de la notation BPMN (Business Process Model and Notation).

\section{Diagramme BPMN Global}
Le diagramme suivant illustre le flux de bout en bout, de la collecte des données capteurs jusqu'à l'intervention de maintenance.

\begin{figure}[H]
\centering
\begin{tikzpicture}[scale=0.6, transform shape]
    % Swimlanes
    \draw[fill=gray!10] (0,10) rectangle (20,14) node[pos=0.5] {}; \node[rotate=90] at (-0.5,12) {\textbf{Système IIoT}};
    \draw[fill=white] (0,6) rectangle (20,10) node[pos=0.5] {}; \node[rotate=90] at (-0.5,8) {\textbf{Plateforme MANTIS}};
    \draw[fill=gray!10] (0,2) rectangle (20,6) node[pos=0.5] {}; \node[rotate=90] at (-0.5,4) {\textbf{Opérateur}};
    \draw[fill=white] (0,-2) rectangle (20,2) node[pos=0.5] {}; \node[rotate=90] at (-0.5,0) {\textbf{Technicien}};

    % Nodes
    \node[circle, draw, thick, fill=green!20] (start) at (1,12) {Start};
    \node[rectangle, draw, rounded corners, fill=blue!20] (collect) at (4,12) {Collecte Données};
    \node[rectangle, draw, rounded corners, fill=orange!20] (ingest) at (4,8) {Ingestion & Traitement};
    \node[rectangle, draw, rounded corners, fill=orange!20] (predict) at (8,8) {Prédiction RUL};
    \node[diamond, draw, fill=yellow!20, aspect=2] (check) at (12,8) {Anomalie ?};
    \node[rectangle, draw, rounded corners, fill=red!20] (alert) at (12,4) {Réception Alerte};
    \node[rectangle, draw, rounded corners, fill=red!20] (analyze) at (16,4) {Analyse Expert};
    \node[diamond, draw, fill=yellow!20, aspect=2] (decision) at (16,0) {Intervention ?};
    \node[rectangle, draw, rounded corners, fill=green!20] (maintain) at (12,0) {Maintenance};
    \node[circle, draw, thick, fill=red!20] (end) at (8,0) {End};

    % Edges
    \draw[->, thick] (start) -- (collect);
    \draw[->, thick] (collect) -- (ingest);
    \draw[->, thick] (ingest) -- (predict);
    \draw[->, thick] (predict) -- (check);
    \draw[->, thick] (check) -- node[above] {Oui} (alert);
    \draw[->, thick] (check) -- node[above] {Non} (predict);
    \draw[->, thick] (alert) -- (analyze);
    \draw[->, thick] (analyze) -- (decision);
    \draw[->, thick] (decision) -- node[above] {Oui} (maintain);
    \draw[->, thick] (decision) -- node[below] {Non} (end);
    \draw[->, thick] (maintain) -- (end);
\end{tikzpicture}
\caption{Processus de Maintenance Prédictive (BPMN)}
\label{fig:bpmn_process}
\end{figure}

\section{Description Détaillée}
\begin{enumerate}
    \item \textbf{Acquisition} : Les capteurs sur les machines industrielles transmettent les données de télémétrie (vibrations, température, pression) en temps réel.
    \item \textbf{Traitement} : La plateforme ingère ces données, les nettoie et calcule des indicateurs de santé (Health Indicators).
    \item \textbf{Analyse} : Les modèles d'IA analysent les flux pour détecter des anomalies ou prédire une défaillance future (RUL).
    \item \textbf{Alerte} : En cas de risque avéré, une notification est envoyée au tableau de bord de l'opérateur.
    \item \textbf{Décision} : L'opérateur valide l'alerte et planifie une intervention technique si nécessaire.
\end{enumerate}

%=============================================================================
% CHAPITRE 2 : ARCHITECTURE MICROSERVICES
%=============================================================================
\chapter{Architecture Microservices}

\section{Vue d'Ensemble (Schéma)}
L'architecture repose sur un modèle événementiel distribué.

\begin{figure}[H]
\centering
\begin{tikzpicture}[scale=0.55, every node/.style={transform shape},
    service/.style={draw, rectangle, rounded corners, minimum width=2.8cm, minimum height=1.3cm, align=center, font=\small},
    db/.style={draw, cylinder, shape border rotate=90, aspect=0.25, minimum width=2cm, minimum height=1cm, align=center, font=\scriptsize}]
    
    % Services
    \node[service, fill=mantisorange!30] (ingestion) at (0,8) {Ingestion};
    \node[service, fill=mantisorange!30] (preproc) at (4,8) {Preprocessing};
    \node[service, fill=mantisorange!30] (predict) at (8,8) {Prediction};
    \node[service, fill=mantisorange!30] (anomaly) at (12,8) {Anomaly};
    \node[service, fill=mantisorange!30] (notif) at (8,4) {Notification};
    
    % Kafka
    \node[service, fill=mantisred!30, minimum width=14cm, minimum height=1cm] (kafka) at (6,6) {Apache Kafka (Event Bus)};
    
    % DBs
    \node[db, fill=mantisblue!20] (redis) at (0,4) {Redis};
    \node[db, fill=mantisblue!20] (feast) at (4,4) {Feast};
    \node[db, fill=mantisblue!20] (tsdb) at (12,4) {TimescaleDB};
    
    % Connections
    \draw[<->, thick] (ingestion) -- (kafka);
    \draw[<->, thick] (preproc) -- (kafka);
    \draw[<->, thick] (predict) -- (kafka);
    \draw[<->, thick] (anomaly) -- (kafka);
    \draw[<->, thick] (notif) -- (kafka);
    
    \draw[<->, dashed] (ingestion) -- (redis);
    \draw[<->, dashed] (preproc) -- (feast);
    \draw[<->, dashed] (predict) -- (tsdb);
    \draw[<->, dashed] (anomaly) -- (tsdb);
    
\end{tikzpicture}
\caption{Architecture Microservices Simplifiée}
\end{figure}

\section{Détail des Microservices}

\subsection{Service Ingestion}
\begin{itemize}
    \item \textbf{Rôle} : Collecte et standardisation des données hétérogènes (OPC UA, MQTT).
    \item \textbf{Technologies} : Python, FastAPI, AsyncIO.
    \item \textbf{Base de données} : Redis (Buffer temporaire).
    \item \textbf{Communication} :
    \begin{itemize}
        \item \textit{Synchrone} : Polling des automates (Modbus/OPC UA).
        \item \textit{Asynchrone} : Publication vers Kafka (Topic: \texttt{raw-data}).
    \end{itemize}
\end{itemize}

\subsection{Service Preprocessing}
\begin{itemize}
    \item \textbf{Rôle} : Nettoyage, normalisation et extraction de features.
    \item \textbf{Technologies} : Python, Pandas, Scikit-learn.
    \item \textbf{Base de données} : Feast (Feature Store).
    \item \textbf{Communication} :
    \begin{itemize}
        \item \textit{Asynchrone} : Consommation/Publication Kafka.
    \end{itemize}
\end{itemize}

\subsection{Service Prediction}
\begin{itemize}
    \item \textbf{Rôle} : Estimation du RUL (Remaining Useful Life) via Deep Learning.
    \item \textbf{Technologies} : PyTorch, LSTM, ONNX Runtime.
    \item \textbf{Base de données} : TimescaleDB (Stockage prédictions), MLflow (Modèles).
    \item \textbf{Communication} :
    \begin{itemize}
        \item \textit{Asynchrone} : Kafka (Stream processing).
        \item \textit{Synchrone} : API REST pour inférence unitaire.
    \end{itemize}
\end{itemize}

\subsection{Service Notification}
\begin{itemize}
    \item \textbf{Rôle} : Diffusion des alertes aux utilisateurs.
    \item \textbf{Technologies} : Node.js ou Python, WebSockets.
    \item \textbf{Base de données} : PostgreSQL (Préférences utilisateurs).
    \item \textbf{Communication} :
    \begin{itemize}
        \item \textit{Asynchrone} : Consommation Kafka.
        \item \textit{Synchrone} : Push WebSocket vers Frontend.
    \end{itemize}
\end{itemize}

%=============================================================================
% CHAPITRE 3 : CONCEPTION DÉTAILLÉE
%=============================================================================
\chapter{Conception de Chaque Microservice}

\section{Service Ingestion}

\subsection{Cas d'Utilisation}
\begin{itemize}
    \item \textbf{Acteur} : Automate Industriel, Système Externe.
    \item \textbf{Scénario} : Connexion à la source, lecture des registres, conversion JSON, envoi au bus.
\end{itemize}

\subsection{Diagramme de Classes}
\begin{figure}[H]
\centering
\begin{tikzpicture}[scale=0.7, transform shape]
    \node[rectangle, draw, align=center] (Service) at (0,0) {\textbf{IngestionService} \\ + start() \\ + stop()};
    \node[rectangle, draw, align=center] (Connector) at (0,-3) {\textbf{<<Interface>> Connector} \\ + connect() \\ + read()};
    \node[rectangle, draw, align=center] (Mqtt) at (-3,-5) {\textbf{MqttConnector}};
    \node[rectangle, draw, align=center] (Opc) at (3,-5) {\textbf{OpcUaConnector}};
    
    \draw[->] (Service) -- (Connector);
    \draw[dashed, ->] (Mqtt) -- (Connector);
    \draw[dashed, ->] (Opc) -- (Connector);
\end{tikzpicture}
\caption{Classes - Ingestion}
\end{figure}

\section{Service Prediction}

\subsection{Cas d'Utilisation}
\begin{itemize}
    \item \textbf{Acteur} : Système (Event), Data Scientist.
    \item \textbf{Scénario} : Réception fenêtre de données, chargement modèle, inférence, sauvegarde résultat.
\end{itemize}

\subsection{Diagramme de Classes}
\begin{figure}[H]
\centering
\begin{tikzpicture}[scale=0.7, transform shape]
    \node[rectangle, draw, align=center] (Service) at (0,0) {\textbf{PredictionService} \\ + predict(data)};
    \node[rectangle, draw, align=center] (Model) at (-4,-2) {\textbf{ModelLoader} \\ + load(version)};
    \node[rectangle, draw, align=center] (Repo) at (4,-2) {\textbf{PredictionRepo} \\ + save(result)};
    
    \draw[->] (Service) -- (Model);
    \draw[->] (Service) -- (Repo);
\end{tikzpicture}
\caption{Classes - Prediction}
\end{figure}

%=============================================================================
% CHAPITRE 4 : MAQUETTES UI/UX (FIGMA)
%=============================================================================
\chapter{Maquettes UI/UX}

\section{Dashboard Principal}
Vue synthétique de l'état du parc.

\begin{figure}[H]
\centering
\begin{tikzpicture}[scale=0.5]
    \draw[fill=white] (0,0) rectangle (16,10);
    \draw[fill=mantisblue] (0,9) rectangle (16,10);
    \node[white] at (2,9.5) {MANTIS Dashboard};
    
    % Sidebar
    \draw[fill=gray!10] (0,0) rectangle (3,9);
    
    % Cards
    \draw[fill=green!20] (4,6) rectangle (7,8); \node at (5.5,7) {Santé: 98\%};
    \draw[fill=red!20] (8,6) rectangle (11,8); \node at (9.5,7) {Alertes: 2};
    \draw[fill=blue!20] (12,6) rectangle (15,8); \node at (13.5,7) {Total: 45};
    
    % Graph
    \draw[draw=gray] (4,1) rectangle (15,5);
    \node at (9.5,3) {[Graphique Temps Réel]};
\end{tikzpicture}
\caption{Maquette Dashboard (Style Figma)}
\end{figure}

\section{Vue Détail Machine}
Visualisation des capteurs et du RUL spécifique.

\begin{figure}[H]
\centering
\begin{tikzpicture}[scale=0.5]
    \draw[fill=white] (0,0) rectangle (16,10);
    \draw[fill=mantisblue] (0,9) rectangle (16,10);
    \node[white] at (2,9.5) {Détail Machine \#123};
    
    % Gauges
    \draw (4,6) circle (1.5); \node at (4,6) {Temp};
    \draw (8,6) circle (1.5); \node at (8,6) {Vib};
    \draw (12,6) circle (1.5); \node at (12,6) {Press};
    
    % RUL
    \draw[fill=orange!20] (4,2) rectangle (12,4);
    \node at (8,3) {\textbf{RUL Estimé : 145 cycles}};
\end{tikzpicture}
\caption{Maquette Détail (Style Figma)}
\end{figure}

\end{document}
